\documentclass{acm_proc_article-sp}

\begin{document}

\title{An Application of Spatially Correlated Poisson Hidden Markov Models for Online Supervised Learning of Bikeshare Data}

\numberofauthors{1}
\author{
\alignauthor
Walter Dempsey\\
       \affaddr{University of Chicago}\\
       \affaddr{5734 S. University Avenue}\\
       \affaddr{Chicago, IL}\\
       \email{wdempsey@galton.uchicago.edu}
}

\maketitle

\begin{abstract}
\end{abstract}


\terms{Log Gaussian Cox Processes, Hidden Markov Model, Online Learning}

\section{Introduction}

\vspace{0.25cm}
{\bf SECTION OUTLINE}
\begin{enumerate}
\item General explanation of technique
\item Do not focus on issues of the analysis -> City has bikes and we need to predict the number at a station
\item We wish to have informed prediction and probabilities
\item We wish to have a means for prediction in the case (1) a new system, (2) a new station in a city.
\item PHMM Approach -> Worry about spatial and temporal
\item Online Learning System for Updating the Parameters
\item Fast Algorithms -> Leverage Fast Algorithms for fitting the basic models
\end{enumerate}
\vspace{0.5cm}


\section{Poisson Hidden Markov Model}

\vspace{0.25cm}
{\bf SECTION OUTLINE}
\begin{enumerate}
\item Introduce the Basic Poisson HMM Model
\end{enumerate}
\vspace{0.5cm}


\section{The Bikeshare Problem}

\vspace{0.25cm}
{\bf SECTION OUTLINE}
\begin{enumerate}
\item Introduce Bikeshare
\item The counts are multinomial in a manner but highly flat (need to think of them as rates)
\item Define the Dual PHMM process
\item Introduce the Censoring Issue and the Correlation B/W the Processes
\end{enumerate}
\vspace{0.5cm}

\section{Spatial-Temporal PHMM}

\vspace{0.25cm}
{\bf SECTION OUTLINE}
\begin{enumerate}
\item The stations are not in isolation
\item We should leverage information about other stations
\item We use a log-Gaussian Cox Process to do this
\item Problems are things like (1) Takes a while to fit, (2) Our parameters are temporally dependent 
\item How do we choose the covariates?
\end{enumerate}
\vspace{0.5cm}

\section{A Parallel Approximate Estimation Algorithm}

\vspace{0.25cm}
{\bf SECTION OUTLINE}
\begin{enumerate}
\item The Gaussian Process Leads to a Complete Graphical Model
\item However the dependency on the other far away nodes is negligible.
\item Therefore, we can approximate the Gaussian Process by an Approximate Block Diagonal GP
\item Similarity to CRFs (but we have a temporal component)
\item This allows us to fit models separately.
\item Problem is boundaries are not distinct
\item End with discussion of Estimation
\end{enumerate}
\vspace{0.5cm}

\section{Online Learning: Algorithm and Prediction}

\vspace{0.25cm}
{\bf SECTION OUTLINE}
\begin{enumerate}
\item Want a manner to update paramaters in an online setting as the `latest numbers' provide the most updated solutions
\item We want to do this in a very simple way
\item Use the online algorithms and approximate block models
\end{enumerate}
\vspace{0.5cm}

\section{Prediction}
{\bf SECTION OUTLINE}
\begin{enumerate}
\item Simple : Station with data and neighboring stations
\item Add a new Station:   Need to infer the model for that station
\item Cluster Analysis
\item What about a new system?  Assume the covariates are okay, then we still need to do a cluster analysis
\end{enumerate}
\vspace{0.5cm}



\section{Experimental Results}

\vspace{0.25cm}

\subsection{Washington D.C. Bikeshare}
\vspace{0.25cm}
{\bf SECTION OUTLINE}
\begin{enumerate}
\item Explain the Complete Dataset
\item Include Maps
\item Explain how the `data' is common to Chicago given the layout of the city but it will not be common to other cities
\item 
\end{enumerate}
\vspace{0.5cm}

\subsection{Performance Indices}

\vspace{0.25cm}
{\bf SECTION OUTLINE}
\begin{enumerate}
\item How do we measure success?
\item We need good short term (`15 min' predictions) vs long term predictions of counts
\item We also want good prediction of the probability of being empty or full?
\item Need baseline and alternatives for comparison
\end{enumerate}
\vspace{0.5cm}

\section{Summary}


\end{document}
