\documentclass{acm_proc_article-sp}

\begin{document}

\title{Addressing Redistribution in Bikeshare Using Cox Processes and Online Supervised Learning}

\numberofauthors{1}
\author{
\alignauthor
Walter Dempsey\\
       \affaddr{University of Chicago}\\
       \affaddr{5734 S. University Avenue}\\
       \affaddr{Chicago, IL}\\
       \email{wdempsey@galton.uchicago.edu}
}

\maketitle

\begin{abstract}
A bicycle sharing system (bikeshare) is a service in which individuals can check-out bicycles from stations for a short period of time and drop them off at any of the stations in the system. The main purpose of bikeshare is to provide transportation for short trips, and it has become common in many cities including New York, Paris, and Beijing. One of the issues with bikeshare is the possible imbalance of bikes across stations, specifically that some stations may have no bikes to check-out while other stations have no slots for arriving bikes. Many cities employ re-distribution techniques on an ad hoc basis. In order to improve the redistribution process, we must estimate the expected number of bikes at a station at some point in the future, as well as the chance that the station becomes empty or full in that time window. To do this, we model the arrival and departure of bikes as a log Gaussian Cox Process. We build a block approximation of the underlying Gaussian processes by clustering  neighborhoods, which will allow for parallel estimation. Moreover, we build online methods for updating the parameters from streaming station data. In order to assess performance, we apply these techniques to the Washington D.C. bikeshare data. We end with a brief discussion of how our approach can be used in analyzing new stations without sufficient data and as well as new networks/cities. 
\end{abstract}


\terms{Log Gaussian Cox Processes, Block-Approximate Gaussian Processes, Parallel Algorithms, Online Learning}

\section{Introduction}

Bicycle Sharing systems have become ubiqutious over the past several years, providing an alternative mode of transportation and improving the connectivity of their respective cities.  These networks (typically referred to as bikeshare) consist of a number of stations distributed throughout the city with individuals able to check-out bicycles from stations for a short period of time before dropping them off at another station in the system.  A common issue with bikeshare is that traffic patterns commonly result in bike imbalance across stations, with some stations being either empty or full.  We call these {\bf extreme} stations.  This leads to delays as bikeshare users must find alternate stations, while also lowering overall ridership due to reliability concerns. Bikeshare systems tackle this issue by some form of redistribution; however, the methods for redistribution of bikes is ad hoc and may be sub-optimal.

For a given time in the future, we wish to predict the number of bikes at the station as well as the probability of the station becoming extreme at some point in this time window given its current state as well as auxiliary variables.  This will allow us to predict extreme stations as well as provide an estimated optimal number of bikes per station during redistribution to avoid extreme events in the next time window.

We model arrival and departure of bikes at each station as a pair of correlated Poisson Processes.  We propose then modeling the entire bikeshare system as a set of correlated Poisson Processes, where two pairs are correlated based on their spatial proximity.  This set is therefore modeled as a log-Gaussian Cox Process in which we capture station-level, temporal, and spatial random effects.  

This approach is computationally intensive, and therefore we propose approximation methods for estimation, which allow us to run estimation in parallel and provide necessary speedups at the cost of some inexactness.  The bikeshare data analyzed is also streaming, and therefore we provide an online algorithm which updates parameter estimates given the new observations.  This allows us to update models given new data without re-fitting the entire model everytime.  We end with an application of our techniques to Washington D.C. bikeshare data.  We analyze the performance, and provide methods for prediction of new stations given estimation from surrounding stations.


\section{Inhomogenous Poisson Processes}

\vspace{0.25cm}
{\bf SECTION OUTLINE}
\begin{enumerate}
\item Introduce the Basic Poisson HMM Model
\end{enumerate}
\vspace{0.5cm}


\section{The Bikeshare Problem}

\vspace{0.25cm}
{\bf SECTION OUTLINE}
\begin{enumerate}
\item Introduce Bikeshare
\item The counts are multinomial in a manner but highly flat (need to think of them as rates)
\item Define the Dual PHMM process
\item Introduce the Censoring Issue and the Correlation B/W the Processes
\end{enumerate}
\vspace{0.5cm}

\section{Spatial-Temporal PHMM}

\vspace{0.25cm}
{\bf SECTION OUTLINE}
\begin{enumerate}
\item The stations are not in isolation
\item We should leverage information about other stations
\item We use a log-Gaussian Cox Process to do this
\item Problems are things like (1) Takes a while to fit, (2) Our parameters are temporally dependent 
\item How do we choose the covariates?
\end{enumerate}
\vspace{0.5cm}

\section{A Parallel Approximate Estimation Algorithm}

\vspace{0.25cm}
{\bf SECTION OUTLINE}
\begin{enumerate}
\item The Gaussian Process Leads to a Complete Graphical Model
\item However the dependency on the other far away nodes is negligible.
\item Therefore, we can approximate the Gaussian Process by an Approximate Block Diagonal GP
\item Similarity to CRFs (but we have a temporal component)
\item This allows us to fit models separately.
\item Problem is boundaries are not distinct
\item End with discussion of Estimation
\end{enumerate}
\vspace{0.5cm}

\section{Online Learning: Algorithm and Prediction}

\vspace{0.25cm}
{\bf SECTION OUTLINE}
\begin{enumerate}
\item Want a manner to update paramaters in an online setting as the `latest numbers' provide the most updated solutions
\item We want to do this in a very simple way
\item Use the online algorithms and approximate block models
\end{enumerate}
\vspace{0.5cm}

\section{Prediction}
{\bf SECTION OUTLINE}
\begin{enumerate}
\item Simple : Station with data and neighboring stations
\item Add a new Station:   Need to infer the model for that station
\item Cluster Analysis
\item What about a new system?  Assume the covariates are okay, then we still need to do a cluster analysis
\end{enumerate}
\vspace{0.5cm}



\section{Experimental Results}

\vspace{0.25cm}

\subsection{Washington D.C. Bikeshare}
\vspace{0.25cm}
{\bf SECTION OUTLINE}
\begin{enumerate}
\item Explain the Complete Dataset
\item Include Maps
\item Explain how the `data' is common to Chicago given the layout of the city but it will not be common to other cities
\item 
\end{enumerate}
\vspace{0.5cm}

\subsection{Performance Indices}

\vspace{0.25cm}
{\bf SECTION OUTLINE}
\begin{enumerate}
\item How do we measure success?
\item We need good short term (`15 min' predictions) vs long term predictions of counts
\item We also want good prediction of the probability of being empty or full?
\item Need baseline and alternatives for comparison
\end{enumerate}
\vspace{0.5cm}

\section{Summary}


\end{document}
